\documentclass{beamer}
\let\Tiny=\tiny %to avoid warnings related to font size and beamer 
\usetheme{Amsterdam}
\usecolortheme{dolphin}
\usecolortheme{orchid}

\usepackage{fixltx2e} %some latex fixes
\usepackage{microtype} %Subliminal refinements towards typographical perfection
\setlength{\emergencystretch}{2em}
\usepackage{amsmath}
\usepackage{amssymb}
\usepackage{amsthm}
\usepackage{mathrsfs} %Support use of the Raph Smith’s Formal Script font in mathematics
\usepackage[sc]{mathpazo} %mathematical fonts
\usepackage{amsfonts}
\usepackage{color}
\newcommand\todo[1]{\textcolor{red}{(TODO: #1)}}
\newcommand\load{L_{\mathrm{max}}}
\newcommand\numberthis{\addtocounter{equation}{1}\tag{\theequation}}
% within an align* environment, number and label the equation. #1 is the label text to be appended to eqn:
\newcommand\neqn[1]{\numberthis\label{eqn:#1}}

%TODO: why are the following defined when using beamer but not for an article?
%\newtheorem{lemma}{Lemma}
%\newtheorem{corollary}{Corollary}
%\newtheorem{theorem}{Theorem}
\newtheorem{claim}{Claim}
\newtheorem{assumption}{Assumption}
\newtheorem{invariant}{Invariant}

%graphs, pictures
\usepackage{readarray}
\usepackage{tikz}
\usepackage{pgf}
\usepackage{xxcolor}
\usetikzlibrary{positioning}

\usepackage{comment}

\usepackage[style=numeric-comp,backend=bibtex]{biblatex}
\addbibresource{../sources}

%reduce font size of footnotes
\let\oldfootnotesize\footnotesize
\renewcommand*{\footnotesize}{\oldfootnotesize\tiny}

%---------------------------------------------------------------------
%Front matter
%---------------------------------------------------------------------
\title{The Power of Two Random Choices}
%\subtitle{ A follow up seminar to the "Parallel Algorithms" lecture}
\author{Martin Kalany}
\institute
{
  Graduate student in Computer Science\\
  Vienna University of Technology\\
}
\date{\today}

% Create Color definition From Template: 
% #1 template name, 
% #2 foreground color name
% #3 background color name
\newcommand{\ccft}[3]{
\usebeamercolor{#1}
\definecolor{#2}{named}{fg}
\definecolor{#3}{named}{bg}
}
%get some colors from the scheme
\ccft{block body}{myblock body fg}{myblock body bg}
\ccft{block title}{myblock title fg}{myblock title bg}

%---------------------------------------------------------------------
%parameters for balls into bins graphs
%---------------------------------------------------------------------
%scale factor
\newcommand\scalefac{0.55}
%diameter of balls
\newcommand\ballsize{5mm}
%number of bins
\newcommand\nrbins{6}
%space between ball and bin
\newcommand\padding{0.1*\ballsize}
%height of a bin
\newcommand\binheight{5*\balldiameter}
%width of a bin
\newcommand\binwidth{\balldiameter}
%gap from bin to bin
\newcommand\bingap{1.6*\balldiameter}
%color of balls
\tikzstyle{ballstyle} = [ball color=black!30!red]

%---------------------------------------------------------------------
%tikz styles and commands
%---------------------------------------------------------------------
\newcommand\balldiameter{2*\ballsize}

%bin style
\tikzstyle{topflat} = [
	minimum width=(\binwidth+2*\padding)*\scalefac, 
	minimum height=(\binheight+4*\padding)*\scalefac,
	append after command={
    		\pgfextra
        		\fill[fill=myblock body bg] 
        			(\tikzlastnode.north east) [rounded corners] |-
        			(\tikzlastnode.south)  -| 
        			(\tikzlastnode.west) [sharp corners] |- 
        			(\tikzlastnode.north) -- cycle;
	         \draw[rounded corners] 
	         	(\tikzlastnode.north east) |- 
	         	(\tikzlastnode.south) -| 
	         	(\tikzlastnode.north west);
    		\endpgfextra
    }
]    

%draw i'th bin
\newcommand\bin[1]{
	\path node[topflat, xshift=#1*\bingap*\scalefac, above, yshift=-\padding*\scalefac]  {};
}

%draw all bins
\newcommand\bins{
	\foreach \ibin in {1,...,\nrbins}
		\bin{\ibin};
}

%paint a ball in bin #1, position #2 
\newcommand\ball[2]{
	\shade[ballstyle] (#1*\bingap,#2*\balldiameter-\ballsize) circle (\ballsize)
}

%put #2 balls into bin 1 <= #1 <= 6
\newcommand\putinbin[2]{
	\ifnum #2 > 0
		\foreach \nrballs in {1,...,#2}
 			\ball{#1}{\nrballs};
 	\fi
}

%put balls into bins. #i: number of balls to put in i'th bin
\newcounter{index}
\newcommand\balls[1]{%
	\getargsC{#1}%
  	\setcounter{index}{0}%
  	\whiledo{\theindex < \narg}{%
    		\stepcounter{index}%
    		\putinbin{\theindex}{\csname arg\romannumeral\theindex\endcsname}%
  	}%
}

%draw balls and bins. #i: number of balls to put in i'th bin
\newcommand\bab[1]{%
	\bins
	\balls{#1}
}

%---------------------------------------------------------------------
%document
%---------------------------------------------------------------------

% Structure:
% 30 min
% 5 min easy stuff (so that everyone understands)
% 5 min more advanced stuff 
% 15 min difficult stuff (the proof, either of the symmetric or asymmetric allocation scheme using the infamous assumption 1
% 5 min further results and wrap up

\begin{document}
\frame{\titlepage}

\section{Balls into bins}
\begin{frame}
    \frametitle{Table of Contents}
    \tableofcontents
\end{frame}

\begin{frame}
\frametitle{The Problem}
\begin{block}{Balls-into-bins games}
Suppose we have $n$ initially empty bins and need to distribute $m \geq n$ balls among them, as evenly as possible. We are interested in the maximum load $\load$, i.e, how many balls are in the fullest bin?
\end{block}
\pause
Applications:
\begin{itemize}
	\item Hashing
	\item Online load balancing
	\item Emulating PRAMs on DMMs
	\item Low congestion circuit routing
\end{itemize}
\end{frame}

\begin{frame}
\frametitle{Single choice}
\framesubtitle{A basic allocation scheme}
\begin{block}{Single choice strategy}
For each ball, choose one of the $n$ bins uniformly and independently at random.
\end{block}
\bigskip
\begin{center}
\begin{tikzpicture}[scale=\scalefac]
	\bab{2 1 5 2 3 0}
\end{tikzpicture}
\end{center}
\end{frame}

\begin{frame}
\frametitle{Randomized allocation process}

Since we deal with a randomized process, we cannot give upper bounds for the maximum load $\load$ with absolute certainty. 

\bigskip

\begin{block}{Definition}
We say that an event $\mathcal E_n$ occurs \emph{with high probability} (w.h.p.) if $\Pr\left[\mathcal E_n \right] \geq 1 - n^{-\alpha}$ for some $\alpha > 0$.
\end{block}
\end{frame}

\begin{frame}
\frametitle{Single choice: Upper bound for the maximum load}
\begin{claim}
When using the single choice strategy to distribute $m$ balls to $n$ bins, the maximum load will be, w.h.p.,
\begin{align}
\load = 
	\begin{cases}
    \frac{\ln n}{\ln\ln n} + O(1)              & m = n \\
    \frac{m}{n} + \Theta\left(\sqrt{\frac{m\ln n}{n}} \right)              & m > n
    \end{cases}
\end{align}
\end{claim}
\end{frame}

\begin{comment}
\begin{frame}
\frametitle{Applications}
\begin{itemize}
	\item Hashing
	\item Online load balancing
	\item Emulating PRAMs on DMMs
	\item Low congestion circuit routing
\end{itemize}
\end{frame}
\end{comment}

%Table of contents at the start of each section
\AtBeginSection[]
{
  \begin{frame}
    \frametitle{Table of Contents}
    \tableofcontents[currentsection]
  \end{frame}
}

\section{Allocation shemes}
\begin{frame}
\frametitle{BiB Model}
\end{frame}

\begin{frame}
\frametitle{Multiple choice strategies}
\end{frame}

\begin{frame}
\frametitle{Greedy}
\end{frame}

\begin{frame}
\frametitle{Always-go-left}
\end{frame}

\begin{frame}
\frametitle{Proof of either Greedy or Always-go-left}
\end{frame}

\begin{comment}
\section{Test}
\begin{frame}
\frametitle{Ignore the following}
\begin{tikzpicture}[scale=\scalefac]
	\bab{1 2 3 4 5 6}
\end{tikzpicture}
\end{frame}
\end{comment}

%---------------------------------------------------------------------
%bibliography
%---------------------------------------------------------------------
\begin{frame}[allowframebreaks]
\frametitle<presentation>{Literature}    
\printbibliography
\end{frame} 	 
\end{document}