\documentclass{acm_proc_article-sp}

\usepackage{comment}
\usepackage{hyperref}
\usepackage{amssymb}

\newcommand*{\PH}{\makebox[1ex]{\textbf{$\cdot$}}}

%\newtheorem{lemma}{Lemma}
%\newtheorem{corollary}[lemma]{Corollary}
%\newtheorem{theorem}[lemma]{Theorem}

\begin{document}

\title{The Power of Two Random Choices}
\subtitle{\large A follow up seminar to the lecture "Parallel Algorithms"\\ by Jesper Larsson Tr\"aff and Francesco Versaci, 2012 winter term at TU Wien.}

\numberofauthors{1}
\author{
\alignauthor
Martin Kalany\\
       0825673
}    
\date{\today}

\maketitle
\begin{abstract}

\end{abstract}


\section{Introduction}
\label{sec:intro}

\subsection{Notation}
\label{sec:notation}
Throughout this paper the number of balls is denoted by $m$ and the number of bins by $n$. An event $\mathcal E$ occurs with high probability if $\Pr\left[\mathcal E \right] = 1 - o(1)$. Asymptotic notations ($O(\PH)$, $o(\PH)$ and $\omega(\PH)$) are always with respect to $n$. Let $f$, $g$ be arbitrary functions. Then, $f \ll g$ means $f = o(g)$ and $ f \gg g$ means $f = \omega(g)$.

\subsection{Balls into Bins}
\label{sec:ballsIntoBins}
Suppose that $m$ balls are sequentially thrown into $n$ bins, where each ball is placed in an independently and uniformly chosen random bin. We are interested in the \emph{number of balls in the fullest bin}, or equivalently the \emph{gap} of the allocation, that is the difference of the number of balls in the fullest bin and the average amount of balls in the bins. 

These \emph{balls into bins games} or \emph{allocation problems} have been studied extensively in the probability literature (see e.g., \cite{JK77}). For the case of $m = n$, the fullest bin contains $\Theta \left( \frac{\log n}{\log \log n} \right)$ \cite{RS98} with high probability. Since the average load of the bins is 1, this is also a bound for the gap. %More accurately, the number of balls in the bin with the most balls is $\Gamma^{-1}\left(n\right)\left(1+O\left(\frac{1}{\log \Gamma^{-1}\left(n\right)}\right)\right)$ \cite{G91}. %TODO: really include Gonnet? 
For the case $m \gg n$ the number of balls in the fullest bin is $\frac{m}{n} + \Theta\left(\sqrt{\frac{m \log n}{n}}\right)$ with high probability.

Consider a variation of the above \emph{balls-into-bins} problem, which we will call \emph{d-choice balls into bins}: For each ball, chose $d \geq 2$ bins independently and uniformly at random and place the ball into the \emph{least full bin}\footnote{Assuming a fixed ordering of the bins, a ball will be put into the left-most bin in case of a tie}. In this variant, the number of balls in the fullest bin is $\frac{\log \log n}{log d} \left( 1 + O\left(1\right)\right)$ for all $m$ with high probability.

 

\subsection{Applications}
\label{sec:applications}


%bibliography
\bibliographystyle{abbrv}
\bibliography{../sources} 
\end{document}
