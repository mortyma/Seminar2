\documentclass{acm_proc_article-sp}

\usepackage{comment}
\usepackage{hyperref}
\usepackage{amssymb}




\newcommand*{\PH}{\makebox[1ex]{\textbf{$\cdot$}}}

%\newtheorem{lemma}{Lemma}
%\newtheorem{corollary}[lemma]{Corollary}
%\newtheorem{theorem}[lemma]{Theorem}

\begin{document}

\title{The Power of Two Random Choices}
\subtitle{\large A follow up seminar to the lecture "Parallel Algorithms"\\ by Jesper Larsson Tr\"aff and Francesco Versaci, 2012 winter term at TU Wien.}

\numberofauthors{1}
\author{
\alignauthor
Martin Kalany\\
       0825673
}    
\date{\today}

\maketitle
\begin{abstract}

\end{abstract}


\section{Introduction}
\label{sec:intro}

\subsection{Notation}
\label{sec:notation}
Throughout this paper the number of balls is denoted by $m$ and the number of bins by $n$. We say that an event $\mathcal E$ occurs \emph{with high probability} (w.h.p.) if $\Pr\left[\mathcal E \right] = 1 - o(1)$, except when stated otherwise. Asymptotic notations ($O(\PH)$, $o(\PH)$ and $\omega(\PH)$) are always with respect to $n$. Let $f$, $g$ be arbitrary functions. Then, $f \ll g$ means $f = o(g)$ and $ f \gg g$ means $f = \omega(g)$.

\subsection{Balls into Bins}
\label{sec:ballsIntoBins}
Suppose that $m$ balls are sequentially thrown into $n$ bins, where each ball is placed in an independently and uniformly chosen random bin. We are interested in the \emph{number of balls in the fullest bin}, also called the \emph{maximum load}. Equivalently, one can state the \emph{gap} of the allocation, that is the difference of the number of balls in the fullest bin and the average amount of balls in the bins. 

These \emph{balls into bins games} or \emph{allocation problems} have been studied extensively in the probability literature (see e.g., \cite{JK77}). When using the strategy as outlined above (that is, choose a bin independently and uniformly at random for each ball), the fullest bin contains $\Theta \left( \frac{\log n}{\log \log n} \right)$ for the case of $m = n$ with high probability \cite{RS98}. Since the average load of the bins is 1, this is also a bound for the gap. %More accurately, the number of balls in the bin with the most balls is $\Gamma^{-1}\left(n\right)\left(1+O\left(\frac{1}{\log \Gamma^{-1}\left(n\right)}\right)\right)$ \cite{G91}. %TODO: really include Gonnet? 
For the \emph{heavily loaded case}, where $m \gg n \log n$ the maximum load is $\frac{m}{n} + \Theta\left(\sqrt{\frac{m \log n}{n}}\right)$ with high probability \cite{RS98}. In the remainder of this paper, "heavily loaded case" will denote the case where $m = \omega\left(n\right)$ and the discussed strategy will be called \emph{once-choice balls into bins}.

Consider a variation of the above \emph{once-choice balls into bins} strategy, which we will call \emph{multiple-choice balls into bins}: For each ball, chose $d \geq 2$ bins independently and uniformly at random and place the ball into the \emph{least full bin}\footnote{If a tie occurs, chose one of the least full bins arbitrarily.}. Note that the $d$ selected bins are not necessarily distinct. In this variant, the maximum load is $\frac{m}{n} + \frac{\log \log n}{log d} \left( 1 + O\left(1\right)\right)$ for all $m$ with high probability. This was first proven by Azar et al. \cite{ABKU99} for the case $m = n$. (However, the case $d = 2$ was implicitly proven by Karp et al. \cite{KLM92}). See also \cite{RMS01} for a simplified proof and an overview of proof techniques for the given problem.  Unfortunately, the analysis breaks down for the heavily loaded case, where $m = \omega\left(n\right)$. Berenbrink at. all \cite{BFZR08} were the first to proof that the same bound holds for the heavily loaded case\footnote{However, they say that an event $\mathcal E$ occurs with high probability if $\Pr\left[\mathcal E \right]  \geq 1- n^{-\alpha}$, for an arbitrarily chosen constant $\alpha \geq 1$.}. A simplified proof may be found in \cite{TW13}. 

The result stated above implies a gap of $\Theta\left(\log \log n \right)$ for arbitrarily large $m$. This is remarkable in several ways:
\begin{itemize}
\item The apparently small change made for the \emph{multiple-choice balls into bins} strategy results in an exponential decrease of the gap between the fullest box and the average load, even for $d=2$. 
\item Each additional choice ($d > 2$) decreases the gap by only a constant factor \cite{RMS01}. 
\item For the \emph{heavily loaded case}, the resulting gap of the \emph{multiple-choice balls into bins} strategy does not depend on the number of balls $m$. In contrast, the bound for the \emph{one-choice} diverges with $m$.
\item The given bounds are \emph{tight}, meaning that no other strategy that places each ball into one of $d$ randomly selected bins achieves a gap that is asymptotically lower.
\end{itemize}

\subsection{Applications}
\label{sec:applications}
\begin{enumerate}
\item Hashing
\item On-line load balancing
\item Dynamic resource allocation
\item Shared memory emulation on distributed memory machines
\end{enumerate}


%bibliography
\bibliographystyle{abbrv}
\bibliography{../sources} 
\end{document}
